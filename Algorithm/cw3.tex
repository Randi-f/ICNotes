% Compile this file by running 
%   pdflatex cw
% and open the generated pdf to see the result.

% Start of document source.
\documentclass{article}

% Add further packages here as required.
\usepackage[a4paper]{geometry}
\usepackage{amsmath}
\usepackage{algpseudocode}

% Start of document preamble.
\title{
  70087 Algorithms \\
  Assessed Coursework
}

\author{Shihan Fu}

% Start of main content.
\begin{document}

% Add preformatted heading.
\maketitle

% Starts a numbered list.
\begin{enumerate}
  % An item in the numbered list.
  \item Answer to Question 1.
    \begin{algorithmic}[1]
      \Statex
      \Procedure{COUNT\_SORTED}{$A ,B$}
        \State $N \gets A.length$
        \For {$i \gets 1$ to $N$}
          \State $dp[i] \gets 1$
        \EndFor
        \For {$i \gets 1$ to $N$}
          \For {$j \gets i-1$ to $0$}
            \If {$A[i]  \textgreater  A[j]$}
              \State $sp[i] \gets Math.max(dp[i],dp[j]+1)$
            \EndIf
          \EndFor
        \EndFor
        \State \Return MAX(dp[N])
      \EndProcedure
    \end{algorithmic}
  % Second item in main list.
  \item Answer to Question 2.
  If my procedure could not use dynamic programming, the time complexity will increase greatly because each subproblems will need to be computed recursively. This may result in an exponential result.
  
  For the worst case, to get dp[N],  we need to compute and compare every dp[i] where i between 1 to N.
  
  In this case, T(N) = $\Theta$(1)  if N = 0
  
  T(N) = T(N-1) + T(N-2) + ... + T(0) + Nc + d    if N $\textgreater$ 1
  
  So, T(N-1) = T(N-2) + ... + T(0) + (N-1)c + d   if N $\textgreater$ 1
  
  In this way, we can get
  
  T(N) = 2T(N-1) + c    if N $\textgreater$ 1
  
  T(N) = 4T(N-2) + 3c   if N $\textgreater$ 2
  
  ...
  
  T(N) = $2^iT(N-i)$ + $(2^i -1)c$  if N $\textgreater$ i where 1 $\leq$ $i < N$
  
  For i = N - 1, we have T(N) = $2^{N-1}T(1)$ + $(2^{N-1} -1)c$
  
  T(1) = T(0) + c + d
  
  So T(N) = $2^{N-1}T(0)$ + $2^{N-1}c$ + $(2^{N-1}-1)c$ + $2^{N-1}d$
  
  therefore, T(N) = $(2^N)c$ + $2^{N-1}T(0)$ + $2^{N-1}d$ -c = $\Theta$$(2^N)$
  
  For the best case, we assume that the array is in an decreasing order. 
  The result of each comparision will be false and each dp[i] will be 1.
  
  In this case, T(N) = $\Theta$(1)  if N = 0
  
  T(N) = Nc     if N $\textgreater$ 1
  
  Therefore, T(N) = $\Theta$(N)
  
  Hence, in total, the upper bound is $O(2^N)$, and the lower bound is $\Omega(N)$.


% End of main list.
\end{enumerate}

\end{document}

